\section{无穷级数}\label{sec:infinite-series}

\subsection{无穷级数的定义}\label{subsec:infinite-series-definition}

\begin{definition}[无穷级数的定义]
  设 $\{a_n\}_{n=1}^{\infty}$ 是一个数列,称 $\sum_{n=1}^{\infty} a_n$ 为无穷级数,或简称为级数。
  级数的部分和定义为
  \begin{equation*}
    S_n = \sum_{k=1}^{n} a_k = a_1 + a_2 + \cdots + a_n.
  \end{equation*}
  如果 $\lim_{n \to \infty} S_n = S$ 存在,则称级数收敛,$S$ 称为级数的和;
  如果不存在,则称级数发散。
\end{definition}

从定义中容易得到一个推论:

\begin{lemma}\label{lem:infinite-series-convergence}
  设 $\{a_n\}_{n=1}^{\infty}$ 是一个数列,$S_n = \sum_{k=1}^{n} a_k$。
  如果 $\lim_{n \to \infty} S_n = S$ 存在,则 $\lim_{n \to \infty} a_n = 0$。
\end{lemma}

从极限的线性性,容易得到\textbf{无穷级数的线性性}:

\begin{property}[无穷级数的线性性]\label{prop:infinite-series-linearity}
  设 $\{a_n\}_{n=1}^{\infty}$ 和 $\{b_n\}_{n=1}^{\infty}$ 是两个数列,$S_n = \sum_{k=1}^{n} a_k$,$T_n = \sum_{k=1}^{n} b_k$。
  如果 $\lim_{n \to \infty} S_n = S$ 和 $\lim_{n \to \infty} T_n = T$ 存在,则
  \begin{equation*}
    \lim_{n \to \infty} (S_n + T_n) = S + T,
  \end{equation*}
  \begin{equation*}
    \lim_{n \to \infty} (c S_n) = c S,
  \end{equation*}
  对任意常数 $c$ 成立。
\end{property}

为方便讨论,定义\textbf{余项}为

\begin{equation*}
  R_n = S - S_n = \sum_{k=n+1}^{\infty} a_k.
\end{equation*}

接下来,有定理:

\begin{theorem}[级数的结合律]\label{thm:infinite-series-associativity}
  设级数 $\sum_{n=1}^{\infty} x_n$ 收敛,则任意添加括号后所得级数仍收敛到原级数的和。
\end{theorem}

\begin{proof}
  设添加括号后每一部分为 $y_n$ ,设 $\{x_n\},\ \{y_n\}$ 的部分和分别为 $S_n$ 和 $T_n$。
  则显然 $T_n$ 是 $S_n$ 的一个子列。
  由 Bolzano-Weierstrass 定理,$T_n$ 收敛到 $S$。
\end{proof}

\subsection{正项级数的收敛性}\label{subsec:positive-series-convergence}

\textbf{正项级数}是指所有项均为正数的级数,即 $\forall n \in \mathbb{N}^+, a_n \geq 0$。
若不等号严格成立,则称为\textbf{严格正项级数}。

显然有结论:

\begin{lemma}\label{lem:positive-series-convergence}
  设 $\{a_n\}_{n=1}^{\infty}$ 是一个正项级数,$S_n = \sum_{k=1}^{n} a_k$。
  则 $S_n$ 收敛当且仅当 $S_n$ 有上界。
\end{lemma}

先讨论两个常见的正项级数:\textbf{几何级数}和\textbf{p-级数}。

几何级数是指形如 $\sum_{n=0}^{\infty} ar^n$ 的级数,其中 $a$ 是常数,$r$ 是常数。
显然,几何级数的收敛性与 $r$ 的大小有关:

\begin{enumerate}
  \item 如果 $|r| < 1$,则几何级数收敛,且其和为 $\frac{a}{1 - r}$;
  \item 如果 $|r| \geq 1$,则几何级数发散。
\end{enumerate}

然后讨论另一个级数\textbf{p-级数}。

\begin{definition}[p-级数]\label{def:p-series}
  设 $p$ 是一个实数,$p > 0$,则称 $\sum_{n=1}^{\infty} \frac{1}{n^p}$ 为 p-级数。
\end{definition}

它的敛散性与 $p$ 的大小有关:

\begin{theorem}[p-级数的敛散性]\label{thm:p-series-convergence}
  设 $p$ 是一个实数,$p > 0$,则:
  \begin{enumerate}
    \item 如果 $p \leq 1$,则 $\sum_{n=1}^{\infty} \frac{1}{n^p}$ 发散;
    \item 如果 $p > 1$,则 $\sum_{n=1}^{\infty} \frac{1}{n^p}$ 收敛。
  \end{enumerate}

\end{theorem}

\begin{proof}
  设 $S_n = \sum_{k=1}^{n} \frac{1}{k^p}$。
  分类讨论。

  当 $p > 1$ 时,按 $\frac{1}{2^p}$ 分组,得到:

  \begin{equation*}
    \begin{aligned}
      \sum_{n=1}^{\infty} \frac{1}{n^p}
      &= \frac{1}{1^p}
      + \left(\frac{1}{2^p} + \frac{1}{3^p}\right)
      + \left(\frac{1}{4^p} + \frac{1}{5^p} + \frac{1}{6^p} + \frac{1}{7^p}\right)
      + \cdots \\
      &< \frac{1}{1^p} + \frac{2}{2^p} + \frac{4}{4^p} + \cdots \\
      &= \sum_{k=0}^{\infty} \left(\frac{1}{2^{p-1}}\right)^k
    \end{aligned}
  \end{equation*}

  从而 $S_n$ 有界,从而收敛。

  当 $p \leq 1$ 时,按 $\frac{1}{2^p+1}$ 分组,得到:

  \begin{equation*}
    \begin{aligned}
      \sum_{n=1}^{\infty} \frac{1}{n^p}
      &= \frac{1}{1^p} + \frac{1}{2^p}
      + \left(\frac{1}{3^p} + \frac{1}{4^p}\right)
      + \left(\frac{1}{5^p} + \frac{1}{6^p} + \frac{1}{7^p} + \frac{1}{8^p}\right)
      + \cdots \\
      &> \frac{1}{1^p} + \frac{1}{2^p} + \frac{2}{3^p} + \frac{4}{5^p} + \cdots \\
      &> \frac{1}{1} + \frac{1}{2} + \frac{2}{3} + \frac{4}{5} + \cdots \\
      &> 1 + \frac{1}{2} + \frac{1}{2} + \frac{1}{2} + \cdots \\
      &= \infty
    \end{aligned}
  \end{equation*}

  从而发散。

\end{proof}

接下来讨论一般的正项级数的敛散性判定。
首先,作为对极限最本质的刻画,有 Cauchy 收敛准则:

\begin{theorem}[Cauchy 收敛准则]\label{thm:cauchy-convergence}
  设 $\{a_n\}_{n=1}^{\infty}$ 是一个正项级数,$S_n = \sum_{k=1}^{n} a_k$。
  则 $S_n$ 收敛当且仅当对于任意 $\varepsilon > 0$,存在正整数 $N$,使得对于任意 $m, n > N$,都有
  \begin{equation*}
    S_m - S_n < \varepsilon.
  \end{equation*}
\end{theorem}

此外,有\textbf{比较判别法}:

\begin{theorem}[比较判别法]\label{thm:comparison-test}
  设 $\{a_n\}_{n=1}^{\infty}$ 和 $\{b_n\}_{n=1}^{\infty}$ 是两个正项级数,$S_n = \sum_{k=1}^{n} a_k$,$T_n = \sum_{k=1}^{n} b_k$。
  如果存在正整数 $N$,使得对于任意 $n > N$,都有 $a_n \leq b_n$,则
  \begin{enumerate}
    \item 如果 $T_n$ 收敛,则 $S_n$ 收敛;
    \item 如果 $S_n$ 发散,则 $T_n$ 发散。
  \end{enumerate}
  此外,有\textbf{极限形式的比较判别法}:
  令 $\lim_{n \to \infty} \frac{a_n}{b_n} = L$,则
  \begin{enumerate}
    \item 若 $0 \leq L < \infty$ ,则 $S_n$ 和 $T_n$ 同收敛
    \item 若 $0 < L \leq \infty$,则 $S_n$ 和 $T_n$ 同发散。
  \end{enumerate}
\end{theorem}

那么自然的可以想到:利用一个容易判定敛散性的级数来判定另一个级数的敛散性。
容易想到一个容易判定敛散性的级数是几何级数。
从而,有 \textbf{Cauchy 比较判别法} 和 \textbf{d'Alembert 判别法}:

\begin{theorem}[Cauchy 比较判别法]\label{thm:cauchy-comparison-test}
  设 $\{a_n\}_{n=1}^{\infty}$ 是一个正项级数,$S_n = \sum_{k=1}^{n} a_k$。
  $r = \limsup_{n \to \infty} \sqrt[n]{a_n}$。
  则有:

  \begin{enumerate}
    \item 若 $r < 1$,则 $S_n$ 收敛;
    \item 若 $r > 1$,则 $S_n$ 发散;
    \item 若 $r = 1$,则 $S_n$ 的敛散性无法用 Cauchy 比较判别法判定。
  \end{enumerate}

\end{theorem}

\begin{proof}
  当 $r < 1$ 时,取 $q$ 使得 $r < q < 1$,则存在正整数 $N$,使得对于任意 $n > N$,都有

  \begin{equation*}
    a_n < q^n.
  \end{equation*}

  由比较判别法可知,$S_n$ 收敛。
\end{proof}

\begin{theorem}[d'Alembert 判别法]\label{thm:d-alembert-comparison-test}
  设 $\{a_n\}_{n=1}^{\infty}$ 是一个正项级数,$S_n = \sum_{k=1}^{n} a_k$。
  则有:
  \begin{enumerate}
    \item 若 $\limsup_{n \to \infty} \frac{a_{n+1}}{a_n} = \overline{r} < 1$,则 $S_n$ 收敛;
    \item 若 $\liminf_{n \to \infty} \frac{a_{n+1}}{a_n} = \underline{r} > 1$,则 $S_n$ 发散;
    \item 若 $\overline{r} \geq 1$ 或 $\underline{r} \leq 1$,则 $S_n$ 的敛散性无法用 d'Alembert 判别法判定。
  \end{enumerate}
\end{theorem}

证明包含在这个引理中:

\begin{lemma}
  设 $\{a_n\}_{n=1}^{\infty}$ 是一个正项级数,则:
  \begin{equation*}
    \liminf_{n \to \infty} \frac{a_{n+1}}{a_n}
    \leq \liminf_{n \to \infty} \sqrt[n]{a_n}
    \leq \limsup_{n \to \infty} \sqrt[n]{a_n}
    \leq \limsup_{n \to \infty} \frac{a_{n+1}}{a_n} .
  \end{equation*}
\end{lemma}

\begin{proof}
  设
  \begin{equation*}
    \overline{r} = \limsup_{n \to \infty} \frac{a_{n+1}}{a_n}
  \end{equation*}
  则:$\forall \varepsilon > 0$,存在正整数 $N$,使得对于任意 $n > N$,都有:
  \begin{equation*}
    \frac{a_{n+1}}{a_n} < \overline{r} + \varepsilon.
  \end{equation*}
  从而:
  \begin{equation*}
    a_n < (\overline{r} + \varepsilon)^{n-N-1} \cdot a_{N+1}
  \end{equation*}
  从而:
  \begin{equation*}
    \limsup_{n \to \infty} \sqrt[n]{a_n}
    \leq \limsup_{n \to \infty} \sqrt[n]{(\overline{r} + \varepsilon)^{n-N-1} \cdot a_{N+1}}
    = \overline{r} + \varepsilon.
  \end{equation*}
  由 $\varepsilon$ 任意小,得:
  \begin{equation*}
    \limsup_{n \to \infty} \sqrt[n]{a_n} \leq \overline{r}.
  \end{equation*}
  另一边的证明同理。
\end{proof}

然而,d'Alembert 判别法和 Cauchy 判别法并不总是适用的。
例如,对于 \textbf{p-级数}:$\sum_{n=1}^{\infty} \frac{1}{n^p}$,两种判别法都无法使用。
因此,给出一些其他的判别法:

\begin{theorem}[Raabe 判别法]\label{thm:raabe-test}
  设 $\{a_n\}_{n=1}^{\infty}$ 是一个正项级数,$S_n = \sum_{k=1}^{n} a_k$。
  令
  \begin{equation*}
    r = \lim_{n \to \infty} n \left( \frac{a_n}{a_{n+1}} - 1\right).
  \end{equation*}
  则有:
  \begin{enumerate}
    \item 若 $r > 1$,则 $S_n$ 收敛;
    \item 若 $r < 1$,则 $S_n$ 发散;
    \item 若 $r = 1$,则 $S_n$ 的敛散性无法用 Raabe 判别法判定。
  \end{enumerate}
\end{theorem}

\begin{proof}
  令 $s>t>1$ ,定义辅助函数 $f(x) = 1 + sx - (1 + x)^t$ 。
  由 $f(0) = 0$ 和 $f'(0) = s - t > 0$ 可知,$\exists \delta > 0, \forall x \in (0, \delta)$,都有 $f(x) > 0$。

  $r > 1$ 时,取 $r > s > t > 1$ ,则有: $\exists N > 0$,使得对于任意 $n > N$,都有:

  \begin{equation*}
    \frac{a_n}{a_{n+1}} > 1 + \frac{s}{n} > \left(1 + \frac{1}{n}\right)^t = \frac{(n+1)^t}{n^t}
  \end{equation*}

  从而 $\{n^t a_n\}$ 从某一项起单调递减,从而有界。设有上界 $A$ ,从而有 $a_n \leq \frac{A}{n^t}$。

  又有 $t > 1$ ,所以 $\sum_{n=1}^{\infty} \frac{1}{n^t}$ 收敛。
  从而原级数 $\sum_{n=1}^{\infty} a_n$ 收敛。

  若 $r < 1$,则取 $r < s < t < 1$,同理可得:原级数发散。

\end{proof}

接下来,考虑到求和与定积分的关系,有\textbf{积分判别法}:

\begin{theorem}[积分判别法]\label{thm:integral-test}
  设 $\{u_n\}_{n=1}^{\infty}$ 是一个正项级数,$S_n = \sum_{k=1}^{n} u_k$。
  函数 $f(x)$ 定义于 $[a, +\infty)$ 且 $f(x) \geq 0$ ,且在任意有限区间上 Riemann 可积。
  取一单调增加趋向于 $+\infty$ 的数列 $\{a_n\}$ ,令:$$\int_{a_n}^{a_{n+1}} f(x) \mathrm{d}x = u_n$$

  则有:反常积分 $\int_{a}^{+\infty} f(x) \mathrm{d}x$ 收敛当且仅当级数 $\sum_{n=1}^{\infty} u_n$ 收敛。

  特别地,若 $f(x)$ 在 $[a, +\infty)$ 上单调递减,取 $a_n = n$,则有:
  反常积分 $\int_{a}^{+\infty} f(x) \mathrm{d}x$ 收敛当且仅当级数 $\sum_{n=[a]+1}^{\infty} f(n)$ 收敛。
\end{theorem}

利用积分判别法可以容易地得到 p-级数的敛散性。

\subsection{任意项级数的敛散性}\label{subsec:arbitrary-series-convergence}

任意项级数是指项的符号不一定相同的级数。
在讨论一般的任意项级数前,先讨论\textbf{交错级数}。

\begin{definition}[交错级数]\label{def:alternating-series}
  若级数 $\sum_{n=1}^{\infty} a_n = \sum_{n=1}^{\infty} (-1)^n u_n$ ,其中 $u_n \geq 0$,则称其为\textbf{交错级数}。
\end{definition}

进一步,有 \textbf{Leibniz 级数}:

\begin{definition}[Leibniz 级数]\label{def:leibniz-series}
  设 $\{u_n\}_{n=1}^{\infty}$ 是一个正项级数。
  如果 $u_n$ 单调递减且 $\lim_{n \to \infty} u_n = 0$,则称 $\sum_{n=1}^{\infty} (-1)^n u_n$ 为 Leibniz 级数。
\end{definition}

对于 Leibniz 级数,有 \textbf{Leibniz 判别法}:

\begin{theorem}[Leibniz 判别法]\label{thm:leibniz-test}
  Leibniz 级数必定收敛
\end{theorem}

\begin{proof}
  设 $S_n = \sum_{k=0}^{n} (-1)^k u_k$,则有:
  \begin{align*}
    S_n &= \sum_{k=0}^{n} (-1)^k u_k \\
    &= u_0 - u_1 + u_2 - u_3 + \cdots + (-1)^n u_n \\
    &= u_0 - (u_1 - u_2) - (u_3 - u_4) + \cdots \tag*{(*)}\\
    &= (u_0 - u_1) + (u_2 - u_3) + \cdots \tag*{(**)}
  \end{align*}
  由单调性和 $(*)$ 可知,$S_n$ 是单调递减的;又由 $(**)$ 可知,$S_n$ 是有界的。
  从而 $S_n$ 收敛。
\end{proof}

为了判定更一般的情况,有 \textbf{Dirichlet-Abel 判别法}。
首先有引理:

\begin{lemma}[Abel 变换]\label{lem:abel-transformation}
  设 $\{a_n\},\ \{b_n\}$ 是两个数列,$A_n = \sum_{k=1}^{n} a_k$,$B_n = \sum_{k=1}^{n} b_k$。
  则有:
  \begin{equation*}
    \sum_{k=1}^{n} a_k b_k = A_n b_n - \sum_{k=1}^{n-1} A_k (b_{k+1} - b_k).
  \end{equation*}
\end{lemma}

也叫\textbf{分部求和公式}。证明易证。

由 Abel 变换,有引理:

\begin{lemma}[Abel 引理]\label{lem:abel-lemma}
  令:
  \begin{enumerate}
    \item $\{a_k\}$ 单调
    \item $\{B_k = \sum_{k=1}^n b_k\}$ 有界 $M$
  \end{enumerate}
  则:
  \begin{equation*}
    \left|\ \sum_{k=1}^p a_k b_k\ \right| \leq M (|a_1| + 2|a_p|)
  \end{equation*}
\end{lemma}

\begin{proof}
  由 Abel 变换和单调性,有:
  \begin{align*}
    \left|\ \sum_{k=1}^p a_k b_k \ \right|
    &\leq |\ a_p B_p \ | + \sum_{k=1}^{p-1} |B_k| |a_{k+1} - a_k| \\
    &\leq M \left(\ |a_p| + \sum_{k=1}^{p-1} |a_{k+1} - a_k| \ \right) \\
    &= M \left(\ |a_p| + \left|\ \sum_{k=1}^{p-1} (a_{k+1} - a_k) \ \right| \ \right) \\
    &= M \left(\ |a_p| + |a_p - a_1|\ \right) \\
    &\leq M \left(\ |a_1| + 2|a_p|\ \right) \\
  \end{align*}
\end{proof}

接下来,有\textbf{级数的 Dirichlet-Abel 判别法}:

\begin{theorem}[级数的 D-A 判别法]\label{thm:dirichlet-abel-test}
  若:
  \begin{enumerate}
    \item \textbf{(Abel 条件)} $\{a_n\}$ 单调有界, $\sum_{n=1}^{\infty} b_n$ 收敛;
    \item \textbf{(Dirichlet 条件)} $\{a_n\}$ 单调趋向于0, $\sum_{k=1}^{n} b_k$ 有界。
  \end{enumerate}
  中任一成立,则级数 $\sum_{n=1}^{\infty} a_n b_n$ 收敛。
\end{theorem}

\begin{proof}
  若 Abel 条件成立,则有: $|a_n| \leq M$ 。
  又有: $\sum_{n=1}^{\infty} b_n$ 收敛,由 Cauchy 收敛准则, $\forall \varepsilon > 0,\ \exists N \in \mathbb{N}^+$,使得对于任意 $m, n > N$,都有:
  \begin{equation*}
    \left| \sum_{k=n+1}^{m} b_k \right| < \varepsilon.
  \end{equation*}
  由 Abel 引理,有:
  \begin{equation*}
    \left|\ \sum_{k=n+1}^{m} a_k b_k \ \right|
    < \varepsilon \left( |a_{n+1}| + 2 |a_m| \right)
    < 3M \varepsilon.
  \end{equation*}
  从而收敛。

  若 Dirichlet 条件成立,则有: $\lim_{n \to \infty} a_n = 0$。
  从而, $\forall \varepsilon > 0,\ \exists N \in \mathbb{N}^+$,使得对于任意 $n > N$,都有:
  \begin{equation*}
    |a_n| < \varepsilon.
  \end{equation*}
  设 $B_n = \sum_{k=1}^{n} b_k$ 有界 $M$。
  则有:
  \begin{equation*}
    \left|\ \sum_{k=n+1}^{m} b_k \ \right|
    = \left| B_m - B_n \right|
    < 2M.
  \end{equation*}
  由 Abel 引理,得:
  \begin{equation*}
    \left|\ \sum_{k=n+1}^{m} a_k b_k \ \right|
    < 2M \left( |a_{n+1}| + 2 |a_m| \right)
    < 6M \varepsilon.
  \end{equation*}
  从而收敛。

\end{proof}

\subsection{绝对收敛与条件收敛}\label{subsec:absolute-convergence}

由于正项级数的敛散性相对容易判断,因此自然地可以想到可以利用正项级数的敛散性来判断任意项级数的敛散性。
从而有\textbf{绝对收敛}的概念:

\begin{definition}[绝对收敛]\label{def:absolute-convergence}
  设 $\sum_{n=1}^{\infty} a_n$ 是一个收敛级数,称其为\textbf{绝对收敛},如果 $\sum_{n=1}^{\infty} |a_n|$ 收敛。
  否则,称其为\textbf{条件收敛}。
\end{definition}

绝对收敛的级数一定收敛,但条件收敛的级数不一定收敛。
这是显然的。

接下来,考虑级数的加法交换律。
我们将一个级数经过重排后得到的级数称作\textbf{更序级数}。
自然地,我们将要考虑:一个更序级数的和是否与原级数的和相等。

首先引入记号:

\begin{equation*}
  x_n^+ = \max\{x_n, 0\},\quad x_n^- = \max\{-x_n, 0\}
\end{equation*}

那么,若 $\sum_{n=1}^{\infty} a_n$ 收敛,则显然有: $\sum_{n=1}^{\infty} a_n = \sum_{n=1}^{\infty} a_n^+ - \sum_{n=1}^{\infty} a_n^-$。

从而有定理:

\begin{theorem}
  若 $\sum_{n=1}^{\infty} a_n$ 绝对收敛,则 $\sum_{n=1}^{\infty} a_n^+$ 和 $\sum_{n=1}^{\infty} a_n^-$ 都收敛;
  若条件收敛,则 $\sum_{n=1}^{\infty} a_n^+$ 和 $\sum_{n=1}^{\infty} a_n^-$ 都发散到 $+\infty$。
\end{theorem}

\begin{proof}
  先证绝对收敛的情况。

  显然有: $0 \leq a_n^+ \leq |a_n|$,$0 \leq a_n^- \leq |a_n|$。
  由绝对收敛,立刻有 $\sum_{n=1}^{\infty} a_n^+$ 与 $\sum_{n=1}^{\infty} a_n^-$ 收敛。

  再证条件收敛的情况。考虑反证。

  不妨假设 $\sum_{n=1}^{\infty} a_n^+$ 收敛,那么由:
  $$\sum_{n=1}^{\infty} a_n = \sum_{n=1}^{\infty} a_n^+ - \sum_{n=1}^{\infty} a_n^-$$
  有:$\sum_{n=1}^{\infty} a_n^-$ 也收敛。
  从而 $\sum_{n=1}^{\infty} |a_n|$ 收敛。
  这与条件收敛矛盾。
\end{proof}

接着,可以证明如下关于更序级数的定理:

\begin{theorem}\label{thm:rearrangement-theorem}
  设 $\sum_{n=1}^{\infty} a_n$ 是一个绝对收敛的级数,$S = \sum_{n=1}^{\infty} a_n$。
  则任意更序级数 $\sum_{n=1}^{\infty} b_n$ 都收敛到 $S$。
\end{theorem}

\begin{proof}
  首先假设 $a_n$ 是一个正项级数。

  此时有: $\sum_{k=1}^{n} b_k \leq \sum_{n=1}^{\infty} a_n$ ,有界,从而收敛。
  从而有: $\sum_{n=1}^{\infty} b_n \leq \sum_{n=1}^{\infty} a_n$ 。
  此时又可以将 $\sum_{n=1}^{\infty} a_n$ 看作是 $\sum_{n=1}^{\infty} b_n$ 的更序级数,
  从而有: $\sum_{n=1}^{\infty} b_n \geq \sum_{n=1}^{\infty} a_n$ 。
  因此, $\sum_{n=1}^{\infty} b_n = \sum_{n=1}^{\infty} a_n$ 。

  对于任意项级数,考虑 $\sum_{n=1}^{\infty} a_n = \sum_{n=1}^{\infty} a_n^+ - \sum_{n=1}^{\infty} a_n^-$。
  由上面的定理, $\sum_{n=1}^{\infty} a_n^+$ 和 $\sum_{n=1}^{\infty} a_n^-$ 都收敛。
  从而 $\sum_{n=1}^{\infty} b_n^+$ 和 $\sum_{n=1}^{\infty} b_n^-$ 都收敛。
  从而 $\sum_{n=1}^{\infty} b_n = \sum_{n=1}^{\infty} b_n^+ - \sum_{n=1}^{\infty} b_n^-$ 收敛。
\end{proof}

以及关于条件收敛级数的定理:

\begin{theorem}[Riemann]\label{thm:riemann-rearrangement-theorem}
  设 $\sum_{n=1}^{\infty} a_n$ 是一个条件收敛的级数,$S = \sum_{n=1}^{\infty} a_n$。
  则存在一个更序级数 $\sum_{n=1}^{\infty} b_n$,使得 $\sum_{n=1}^{\infty} b_n$ 收敛到任意 $S' \in \mathbb{R} \cup \{\pm\infty \}$。
\end{theorem}

\begin{proof}
  先考虑 $S' \in \mathbb{R}$ 的情况。

  考虑 $\sum_{n=1}^{\infty} a_n = \sum_{n=1}^{\infty} a_n^+ - \sum_{n=1}^{\infty} a_n^-$。
  由上面的定理, $\sum_{n=1}^{\infty} a_n^+$ 和 $\sum_{n=1}^{\infty} a_n^-$ 都发散到 $+\infty$。

  那么,必然存在一个最小的 $m_1$,使得 $$\sum_{n=1}^{m_1} a_n^+ > S'$$
  且存在一个最小的 $n_1$ ,使得 $$\sum_{n=1}^{m_1} a_n^+ - \sum_{n=1}^{n_1} a_n^- < S'$$
  这样的操作总是可以进行的,因此总是可以得到 $a_n$ 的一个更序级数,使得它的部分和满足:
  $$S'_n \in (S' - x_{n_k}^-,\ S' + x_{m_k}^+)$$

  又: $\sum_{n=1}^{\infty} a_n$ 收敛,从而有:
  $$\lim_{n\to \infty} a_n^+ = \lim_{n\to \infty} a_n^- = 0$$
  从而有: $\sum_{n=1}^{\infty} b_n = S'$。

  然后考虑 $S' = \pm \infty$ 的情况。显然只需考虑 $S' = +\infty$ 的情况。

  由于: $$\lim_{n\to \infty} a_n^+ = \lim_{n\to \infty} a_n^- = 0$$
  从而 $\{a_n^-\}$ 有界。
  令 $M = \lceil \sup \{a_n^-\} \rceil$ ,则:
  $$\exists\ n_k,\ \mathrm{s.t.}\ \sum_{n=1}^{n_k} a_n^+ > k + M \Rightarrow \sum_{n=1}^{n_k} a_n^+ - a_k^- > k$$
  从而可以构造更序级数 $\sum_{n=1}^{\infty} b_n$ ,使得它满足:
  $$\forall\ G>0,\ \exists\ N > 0,\ \forall\ n > N,\ \sum_{k=1}^{n} b_n > G$$
  从而更序级数发散。
\end{proof}

最后讨论级数的乘法。

首先定义\textbf{ Cauchy 乘积}:

\begin{definition}[Cauchy 乘积]\label{def:cauchy-product}
  设 $\sum_{n=1}^{\infty} a_n$ 和 $\sum_{n=1}^{\infty} b_n$ 是两个级数,$S = \sum_{n=1}^{\infty} a_n$,$T = \sum_{n=1}^{\infty} b_n$。
  则称:
  \begin{equation*}
    \prod_{n=1}^{\infty} (1 + a_n) = \prod_{n=1}^{\infty} (1 + b_n) = \prod_{n=1}^{\infty} (1 + c_n)
  \end{equation*}
  为 Cauchy 乘积。
  其中:$c_n = \sum_{k=0}^{n} a_k b_{n-k}$。
\end{definition}

% TODO: 关于级数的乘积的部分

\subsection{无穷乘积}\label{subsec:infinite-product}

% TODO: 关于无穷乘积的部分

\subsection{函数项级数}\label{subsec:function-series}

% TODO: 关于函数项级数的部分
