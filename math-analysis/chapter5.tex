\section{无穷级数}\label{sec:infinite-series}

\subsection{无穷级数的定义}\label{subsec:infinite-series-definition}

\begin{definition}[无穷级数的定义]
  设 $\{a_n\}_{n=1}^{\infty}$ 是一个数列,称 $\sum_{n=1}^{\infty} a_n$ 为无穷级数,或简称为级数。
  级数的部分和定义为
  \begin{equation*}
    S_n = \sum_{k=1}^{n} a_k = a_1 + a_2 + \cdots + a_n.
  \end{equation*}
  如果 $\lim_{n \to \infty} S_n = S$ 存在,则称级数收敛,$S$ 称为级数的和;
  如果不存在,则称级数发散。
\end{definition}

从定义中容易得到一个推论:

\begin{lemma}\label{lem:infinite-series-convergence}
  设 $\{a_n\}_{n=1}^{\infty}$ 是一个数列,$S_n = \sum_{k=1}^{n} a_k$。
  如果 $\lim_{n \to \infty} S_n = S$ 存在,则 $\lim_{n \to \infty} a_n = 0$。
\end{lemma}

从极限的线性性,容易得到\textbf{无穷级数的线性性}:

\begin{property}[无穷级数的线性性]\label{prop:infinite-series-linearity}
  设 $\{a_n\}_{n=1}^{\infty}$ 和 $\{b_n\}_{n=1}^{\infty}$ 是两个数列,$S_n = \sum_{k=1}^{n} a_k$,$T_n = \sum_{k=1}^{n} b_k$。
  如果 $\lim_{n \to \infty} S_n = S$ 和 $\lim_{n \to \infty} T_n = T$ 存在,则
  \begin{equation*}
    \lim_{n \to \infty} (S_n + T_n) = S + T,
  \end{equation*}
  \begin{equation*}
    \lim_{n \to \infty} (c S_n) = c S,
  \end{equation*}
  对任意常数 $c$ 成立。
\end{property}

为方便讨论,定义\textbf{余项}为

\begin{equation*}
  R_n = S - S_n = \sum_{k=n+1}^{\infty} a_k.
\end{equation*}

\subsection{正项级数的收敛性}\label{subsec:positive-series-convergence}

\textbf{正项级数}是指所有项均为正数的级数,即 $\forall n \in \mathbb{N}^+, a_n \geq 0$。
若不等号严格成立,则称为\textbf{严格正项级数}。

显然有结论:

\begin{lemma}\label{lem:positive-series-convergence}
  设 $\{a_n\}_{n=1}^{\infty}$ 是一个正项级数,$S_n = \sum_{k=1}^{n} a_k$。
  则 $S_n$ 收敛当且仅当 $S_n$ 有上界。
\end{lemma}

先讨论两个常见的正项级数:\textbf{几何级数}和\textbf{p-级数}。

几何级数是指形如 $\sum_{n=0}^{\infty} ar^n$ 的级数,其中 $a$ 是常数,$r$ 是常数。
显然,几何级数的收敛性与 $r$ 的大小有关:

\begin{enumerate}
  \item 如果 $|r| < 1$,则几何级数收敛,且其和为 $\frac{a}{1 - r}$;
  \item 如果 $|r| \geq 1$,则几何级数发散。
\end{enumerate}

然后讨论另一个级数\textbf{p-级数}。

\begin{definition}[p-级数]\label{def:p-series}
  设 $p$ 是一个实数,$p > 0$,则称 $\sum_{n=1}^{\infty} \frac{1}{n^p}$ 为 p-级数。
\end{definition}

它的敛散性与 $p$ 的大小有关:

\begin{theorem}[p-级数的敛散性]\label{thm:p-series-convergence}
  设 $p$ 是一个实数,$p > 0$,则:
  \begin{enumerate}
    \item 如果 $p \leq 1$,则 $\sum_{n=1}^{\infty} \frac{1}{n^p}$ 发散;
    \item 如果 $p > 1$,则 $\sum_{n=1}^{\infty} \frac{1}{n^p}$ 收敛。
  \end{enumerate}

\end{theorem}

\begin{proof}
  设 $S_n = \sum_{k=1}^{n} \frac{1}{k^p}$。
  分类讨论。

  当 $p > 1$ 时,按 $\frac{1}{2^p}$ 分组,得到:

  \begin{equation*}
    \begin{aligned}
      \sum_{n=1}^{\infty} \frac{1}{n^p}
      &= \frac{1}{1^p}
      + \left(\frac{1}{2^p} + \frac{1}{3^p}\right)
      + \left(\frac{1}{4^p} + \frac{1}{5^p} + \frac{1}{6^p} + \frac{1}{7^p}\right)
      + \cdots \\
      &< \frac{1}{1^p} + \frac{2}{2^p} + \frac{4}{4^p} + \cdots \\
      &= \sum_{k=0}^{\infty} \left(\frac{1}{2^{p-1}}\right)^k
    \end{aligned}
  \end{equation*}

  从而 $S_n$ 有界,从而收敛。

  当 $p \leq 1$ 时,按 $\frac{1}{2^p+1}$ 分组,得到:

  \begin{equation*}
    \begin{aligned}
      \sum_{n=1}^{\infty} \frac{1}{n^p}
      &= \frac{1}{1^p} + \frac{1}{2^p}
      + \left(\frac{1}{3^p} + \frac{1}{4^p}\right)
      + \left(\frac{1}{5^p} + \frac{1}{6^p} + \frac{1}{7^p} + \frac{1}{8^p}\right)
      + \cdots \\
      &> \frac{1}{1^p} + \frac{1}{2^p} + \frac{2}{3^p} + \frac{4}{5^p} + \cdots \\
      &> \frac{1}{1} + \frac{1}{2} + \frac{2}{3} + \frac{4}{5} + \cdots \\
      &> 1 + \frac{1}{2} + \frac{1}{2} + \frac{1}{2} + \cdots \\
      &= \infty
    \end{aligned}
  \end{equation*}

  从而发散。

\end{proof}

接下来讨论一般的正项级数的敛散性判定。
首先,作为对极限最本质的刻画,有 Cauchy 收敛准则:

\begin{theorem}[Cauchy 收敛准则]\label{thm:cauchy-convergence}
  设 $\{a_n\}_{n=1}^{\infty}$ 是一个正项级数,$S_n = \sum_{k=1}^{n} a_k$。
  则 $S_n$ 收敛当且仅当对于任意 $\varepsilon > 0$,存在正整数 $N$,使得对于任意 $m, n > N$,都有
  \begin{equation*}
    S_m - S_n < \varepsilon.
  \end{equation*}
\end{theorem}

此外,有\textbf{比较判别法}:

\begin{theorem}[比较判别法]\label{thm:comparison-test}
  设 $\{a_n\}_{n=1}^{\infty}$ 和 $\{b_n\}_{n=1}^{\infty}$ 是两个正项级数,$S_n = \sum_{k=1}^{n} a_k$,$T_n = \sum_{k=1}^{n} b_k$。
  如果存在正整数 $N$,使得对于任意 $n > N$,都有 $a_n \leq b_n$,则
  \begin{enumerate}
    \item 如果 $T_n$ 收敛,则 $S_n$ 收敛;
    \item 如果 $S_n$ 发散,则 $T_n$ 发散。
  \end{enumerate}
  此外,有\textbf{极限形式的比较判别法}:
  令 $\lim_{n \to \infty} \frac{a_n}{b_n} = L$,则
  \begin{enumerate}
    \item 若 $0 \leq L < \infty$ ,则 $S_n$ 和 $T_n$ 同收敛
    \item 若 $0 < L \leq \infty$,则 $S_n$ 和 $T_n$ 同发散。
  \end{enumerate}
\end{theorem}

那么自然的可以想到:利用一个容易判定敛散性的级数来判定另一个级数的敛散性。
容易想到一个容易判定敛散性的级数是几何级数。
从而,有 \textbf{Cauchy 比较判别法} 和 \textbf{d'Alembert 判别法}:

\begin{theorem}[Cauchy 比较判别法]\label{thm:cauchy-comparison-test}
  设 $\{a_n\}_{n=1}^{\infty}$ 是一个正项级数,$S_n = \sum_{k=1}^{n} a_k$。
  $r = \limsup_{n \to \infty} \sqrt[n]{a_n}$。
  则有:

  \begin{enumerate}
    \item 若 $r < 1$,则 $S_n$ 收敛;
    \item 若 $r > 1$,则 $S_n$ 发散;
    \item 若 $r = 1$,则 $S_n$ 的敛散性无法用 Cauchy 比较判别法判定。
  \end{enumerate}

\end{theorem}

\begin{proof}
  当 $r < 1$ 时,取 $q$ 使得 $r < q < 1$,则存在正整数 $N$,使得对于任意 $n > N$,都有

  \begin{equation*}
    a_n < q^n.
  \end{equation*}

  由比较判别法可知,$S_n$ 收敛。
\end{proof}

\begin{theorem}[d'Alembert 判别法]\label{thm:d-alembert-comparison-test}
  设 $\{a_n\}_{n=1}^{\infty}$ 是一个正项级数,$S_n = \sum_{k=1}^{n} a_k$。
  则有:
  \begin{enumerate}
    \item 若 $\limsup_{n \to \infty} \frac{a_{n+1}}{a_n} = \overline{r} < 1$,则 $S_n$ 收敛;
    \item 若 $\liminf_{n \to \infty} \frac{a_{n+1}}{a_n} = \underline{r} > 1$,则 $S_n$ 发散;
    \item 若 $\overline{r} \geq 1$ 或 $\underline{r} \leq 1$,则 $S_n$ 的敛散性无法用 d'Alembert 判别法判定。
  \end{enumerate}
\end{theorem}

证明包含在这个引理中:

\begin{lemma}
  设 $\{a_n\}_{n=1}^{\infty}$ 是一个正项级数,则:
  \begin{equation*}
    \liminf_{n \to \infty} \frac{a_{n+1}}{a_n}
    \leq \liminf_{n \to \infty} \sqrt[n]{a_n}
    \leq \limsup_{n \to \infty} \sqrt[n]{a_n}
    \leq \limsup_{n \to \infty} \frac{a_{n+1}}{a_n} .
  \end{equation*}
\end{lemma}

\begin{proof}
  设
  \begin{equation*}
    \overline{r} = \limsup_{n \to \infty} \frac{a_{n+1}}{a_n}
  \end{equation*}
  则:$\forall \varepsilon > 0$,存在正整数 $N$,使得对于任意 $n > N$,都有:
  \begin{equation*}
    \frac{a_{n+1}}{a_n} < \overline{r} + \varepsilon.
  \end{equation*}
  从而:
  \begin{equation*}
    a_n < (\overline{r} + \varepsilon)^{n-N-1} \cdot a_{N+1}
  \end{equation*}
  从而:
  \begin{equation*}
    \limsup_{n \to \infty} \sqrt[n]{a_n}
    \leq \limsup_{n \to \infty} \sqrt[n]{(\overline{r} + \varepsilon)^{n-N-1} \cdot a_{N+1}}
    = \overline{r} + \varepsilon.
  \end{equation*}
  由 $\varepsilon$ 任意小,得:
  \begin{equation*}
    \limsup_{n \to \infty} \sqrt[n]{a_n} \leq \overline{r}.
  \end{equation*}
  另一边的证明同理。
\end{proof}

然而,d'Alembert 判别法和 Cauchy 判别法并不总是适用的。
例如,对于 \textbf{p-级数}:$\sum_{n=1}^{\infty} \frac{1}{n^p}$,两种判别法都无法使用。
因此,给出一些其他的判别法:

\begin{theorem}[Raabe 判别法]\label{thm:raabe-test}
  设 $\{a_n\}_{n=1}^{\infty}$ 是一个正项级数,$S_n = \sum_{k=1}^{n} a_k$。
  令
  \begin{equation*}
    r = \lim_{n \to \infty} n \left( \frac{a_n}{a_{n+1}} - 1\right).
  \end{equation*}
  则有:
  \begin{enumerate}
    \item 若 $r > 1$,则 $S_n$ 收敛;
    \item 若 $r < 1$,则 $S_n$ 发散;
    \item 若 $r = 1$,则 $S_n$ 的敛散性无法用 Raabe 判别法判定。
  \end{enumerate}
\end{theorem}

\begin{proof}
  令 $s>t>1$ ,定义辅助函数 $f(x) = 1 + sx - (1 + x)^t$ 。
  由 $f(0) = 0$ 和 $f'(0) = s - t > 0$ 可知,$\exists \delta > 0, \forall x \in (0, \delta)$,都有 $f(x) > 0$。

  $r > 1$ 时,取 $r > s > t > 1$ ,则有: $\exists N > 0$,使得对于任意 $n > N$,都有:

  \begin{equation*}
    \frac{a_n}{a_{n+1}} > 1 + \frac{s}{n} > \left(1 + \frac{1}{n}\right)^t = \frac{(n+1)^t}{n^t}
  \end{equation*}

  从而 $\{n^t a_n\}$ 从某一项起单调递减,从而有界。设有上界 $A$ ,从而有 $a_n \leq \frac{A}{n^t}$。

  又有 $t > 1$ ,所以 $\sum_{n=1}^{\infty} \frac{1}{n^t}$ 收敛。
  从而原级数 $\sum_{n=1}^{\infty} a_n$ 收敛。

  若 $r < 1$,则取 $r < s < t < 1$,同理可得:原级数发散。

\end{proof}

接下来,考虑到求和与定积分的关系,有\textbf{积分判别法}:

\begin{theorem}[积分判别法]\label{thm:integral-test}
  设 $\{u_n\}_{n=1}^{\infty}$ 是一个正项级数,$S_n = \sum_{k=1}^{n} u_k$。
  函数 $f(x)$ 定义于 $[a, +\infty)$ 且 $f(x) \geq 0$ ,且在任意有限区间上 Riemann 可积。
  取一单调增加趋向于 $+\infty$ 的数列 $\{a_n\}$ ,令:$$\int_{a_n}^{a_{n+1}} f(x) \mathrm{d}x = u_n$$

  则有:反常积分 $\int_{a}^{+\infty} f(x) \mathrm{d}x$ 收敛当且仅当级数 $\sum_{n=1}^{\infty} u_n$ 收敛。

  特别地,若 $f(x)$ 在 $[a, +\infty)$ 上单调递减,取 $a_n = n$,则有:
  反常积分 $\int_{a}^{+\infty} f(x) \mathrm{d}x$ 收敛当且仅当级数 $\sum_{n=[a]+1}^{\infty} f(n)$ 收敛。
\end{theorem}

利用积分判别法可以容易地得到 p-级数的敛散性。

\subsection{任意项级数的敛散性}\label{subsec:arbitrary-series-convergence}

任意项级数是指项的符号不一定相同的级数。
